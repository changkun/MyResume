\section{\textbf{Work Experiences}}
  \resumeSubHeadingListStartNoLabel

    \resumeSubheading
      {Research Assistant}{Munich, Germany}
      {University of Munich}{Aprl. 2018 -- Present}
      \resumeItemListStart
      \resumeItem{Lecturer}{I am responsible for the design and teaching of the \href{http://www.medien.ifi.lmu.de/lehre/ws2021/gp/}{Practical Geometry Processing} course. In the class, I mainly teach students about geometry processing algorithms, practice how to implement them from scratch, and eventually writing reproducible rendering scripts in Blender. Resources are on \href{https://github.com/mimuc/gp-ws2021}{GitHub}.}
      \resumeItem{DevOps}{I am responsible for compatibility development work and the execution of the eventual migration operation of a 15 years old PHP and SVN-based CMS system that was developed in 2005. Resources are on \href{https://github.com/changkun/destrictor}{GitHub}.}
      \resumeItem{Teaching Assistant}{I am responsible for the design and organization of the practical part of the Lecture \href{http://www.medien.ifi.lmu.de/lehre/ss20/cg1/}{Computer Graphics}. Behind the scene, the whole coding exercises are redesigned for the fit of modern topics in graphics. Resources are on \href{https://github.com/mimuc/cg1-ss20}{GitHub}.}
      \resumeItem{Teaching Assistant}{I am one of the responsible people for the design and organization of the practical part of Lecture \href{http://www.medien.ifi.lmu.de/lehre/ws1920/omm/}{Online Multimedia}. Behind the scene, I bring novel web development topics into the teaching, such as React, Docker, Kubernetes, etc. Resources are on \href{https://github.com/mimuc/omm-ws1920}{GitHub}.}
      \resumeItem{Teaching Assistant}{I am responsible for the organization of \href{http://www.medien.ifi.lmu.de/lehre/ws1920/hs/}{Seminar Advances in Computer Graphics}.}
      \resumeItem{Tutor}{\href{http://www.dbs.ifi.lmu.de/cms/studium_lehre/lehre_master/deep1819/index.html}{Deep Learning and Artificial Intelligence}, notes on \href{https://github.com/changkun/ss18-machine-learning-tutorial}{GitHub}}
      \resumeItem{Tutor}{\href{http://www.dbs.ifi.lmu.de/cms/studium_lehre/lehre_master/ml18/index.html}{Machine Learning}, notes on \href{https://github.com/changkun/ws-18-19-deep-learning-tutorial}{GitHub}.}
      \resumeItemListEnd

    \resumeSubheading
      {Backend Software Engineering (Remote)}{Chengdu, China}
      {\href{https://labex.io/}{LabEx Technology Ltd}}{Apr. 2018 -- Jan. 2019}
      \resumeItemListStart
      \resumeItem{Team leader and leading backend development of the oversea product}{
        I lead and responsible for the product development in backend and frontend. 
        I evolve the existing architecture and split a monolithic backend web application 
        into multiple microservices. 
        The product scales machine cluster from 20 to 200 for active daily users, 
        and its user group increases from 5k to 30k during my incumbency.
      }
      \resumeItem{Remote desktop Control Proxy}{
        I responsible and developed a middleware that provides generic remote desktop proxy in Go. 
        The proxy translates VNC/RDP/SSH protocol data, 
        and establish WebSocket connection to a web browser for providing remote desktop GUI.
      }
      \resumeItem{Multi-cloud automation}{
        I developed a fully automated multi-cloud resource management microservice in Go. 
        The service defines a general abstraction cross all cloud provider, 
        it automatically manages all user requested resources allocation and 
        releases outdated resources. 
        For instance, a user of the service can allocate new cloud instances 
        for temporal using without noticing the instance was allocated in either AWS, AlibabaCloud, or others. 
        The service supports more than 15 cloud products and integrated 3 cloud providers, 
        being able to support almost unlimited concurrent users and has been used by 10k+ users.
      }
      \resumeItem{Cluster management service}{
        I developed a microservice in Go that similar to Kubernetes and Docker Swarm. 
        The service manages multiple server clusters, 
        and auto-scaling its cluster size upon request cross multiple cloud providers. 
        Each cluster contains multiple physical machines, 
        and each machine runs many docker containers. 
        The key feature of the service eliminates the difference between the physical machine 
        and the docker container. 
        The runtime of the service includes a system monitor with request prediction algorithm 
        that I invented for efficient auto-scaling with consideration of overcommit ratio 
        and a task scheduler for managing all distributed asynchronous task execution 
        with two-level caching optimization.
      }
      \resumeItem{Used tech. stack}{
        Vue, jQuery, Webpack, Electron; 
        Backend: Go, Cgo, Gin, Beego, gRPC, MySQL, MongoDB, Redis, Hypervisor, Nginx, 
        Docker, Kubernetes, AWS, AlibabaCloud, etc.}
      \resumeItemListEnd

    \resumeSubheading
    {Fullstack Engineer (Freelance)}{Munich, Germany}
    {\href{https://magiclingua.com/}{Rocketlingo UG}}{Nov. 2017 -- Mar. 2018}
    \resumeItemListStart
      \resumeItem{Language Teaching Voice Bot}
        {I am part of the team in developing a voice bot that provides English learning teaching service. 
        The bot can communicate with its user and improve their English skill by the real-time response. 
        My responsibility is to implement the backend support designed conversations using Amazon Alexa.}
      \resumeItem{Speech Recognition Solution \& Web Development}{
        I responsible for the development of speech recognition solution over web technologies, 
        such as using WebSocket for audio streaming, 
        using Google Cloud STT and TTS services for speech recognition and synthesis, etc. 
        The challenging part of using existing speech recognition service for 
        a language learning application is a new language learner sometimes does not produces 
        positive audio samples, and even multilingual. 
        Therefore, I developed many text-based falt tolerances technique 
        for improving the understanding of user speech based on machine learning algorithms.
      }
      \resumeItem{Used Tech. Stack}
        {Frontend: Angular,  
        Backend: NodeJS, ExpressJS, WebSocket, Python, Flask, MongoDB, Elasticsearch, 
        AWS Serverless, Tensorflow, Numpy, Matplotlib}
    \resumeItemListEnd

  \ifthenelse{\boolean{cv}}{

    \resumeSubheading
      {\href{https://shiyanlou.com/}{Shiyanlou}}{Chengdu, China}
      {Software Engineer (Intern)}{Jun. 2016 -- Sep. 2016}
      \resumeItemListStart
        \resumeItem{Recommendation System}
          {Online course recommendation system; Python; MongoDB; Collaborate Filtering}
        \resumeItem{Distributed Log System}
          {Elasticsearch; Logstash; Kibana; Redis}
        \resumeItem{Cross Platform Desktop Client}
          {Electron based development for macOS/Windows/Linux}
        \resumeItem{Teaching Meterial Writing}
          {Modern C++ based teaching meterial writing}
      \resumeItemListEnd

    % \resumeSubheading
    %   {\href{http://www.medien.ifi.lmu.de}{LMU Media Informatics Group}}{Munich, Germany}
    %   {Student Researcher (Intern)}{Oct. 2016 -- Jan. 2016}
    %   \resumeItemListStart
    %     \resumeItem{Research topic}
    %       {Combining motion sensors and touch offset classify hand Posture and users}
    %     \resumeItem{Outcomes}
    %       {A iOS prototype App for User Study, densely data analysis via Support Vector Machine; paper report}
    %   \resumeItemListEnd

    \resumeSubheading
      {Academic Tutor}{Chengdu, China}
      {SWUN Lecture ``Human-Computer Interaction''}{Jul. 2015}
  }{}

  \resumeSubHeadingListEnd